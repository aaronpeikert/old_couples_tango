\documentclass[12pt,t]{beamer}
\usepackage{graphicx}
\usepackage{tikz}
\setbeameroption{hide notes}
\setbeamertemplate{note page}[plain]
\usepackage{listings}

\input{header.tex}

%%%%%%%%%%%%%%%%%%%%%%%%%%%%%%%%%%%%%%%%%%%%%%%%%%%%%%%%%%%%%%%%%%%%%%
% end of header
%%%%%%%%%%%%%%%%%%%%%%%%%%%%%%%%%%%%%%%%%%%%%%%%%%%%%%%%%%%%%%%%%%%%%%

% title info

\title{Biobehavioral Pathways Underlying Spousal Health Dynamics: Its Nature, Correlates, and Consequences}
\author{\href{https://github.com/aaronpeikert/}{Aaron Peikert}}
\institute{Humboldt{\textendash}Universität zu Berlin}
\date{
\scriptsize {\lolit Slides:} \href{https://github.com/aaronpeikert/old_couples_tango}{\tt \scriptsize
  \color{foreground} https://github.com/aaronpeikert/old\_couples\_tango}
}


\begin{document}

% title slide
{
{
\setbeamertemplate{footline}{} % no page number here
\frame{
  \titlepage

  \vfill \hfill \includegraphics[height=6mm]{Figs/cc-zero.png} \vspace*{-1cm}
  
  \note{These Slides are for a short Talk (10min) intended for the seminar: „Individual development across the lifespan“ by Prof. Gersdorf.

    Source: {\tt https://github.com/aaronpeikert/old\_couples\_tango}
}}
}
\begin{frame}[c]{Being smimilar - becoming similar}
  Which other factors could play a role why married couples show similar health behavior? Maybe because similarity leads to attraction which can lead to marriage? In other words: How big is the influence of the actual marriage, how big / is there at all an impact of general similarity between spouses?  
  https://link.springer.com/article/10.1007/s10519-011-9509-7
\end{frame}
\begin{frame}[c]{Gender}
  Is there a gender difference in how stress of the spouse efffects the other spouse? I imagine women are more likely to be affected.
  doi.org/10.1016/j.socscimed.2007.02.007
\end{frame}
\end{document}
