\documentclass[12pt,t]{beamer}
\usepackage{graphicx}
\setbeameroption{hide notes}
\setbeamertemplate{note page}[plain]
\usepackage{listings}

\input{header.tex}

%%%%%%%%%%%%%%%%%%%%%%%%%%%%%%%%%%%%%%%%%%%%%%%%%%%%%%%%%%%%%%%%%%%%%%
% end of header
%%%%%%%%%%%%%%%%%%%%%%%%%%%%%%%%%%%%%%%%%%%%%%%%%%%%%%%%%%%%%%%%%%%%%%

% title info

\title{Forest of Eden}
\author{\href{https://github.com/aaronpeikert/}{Aaron Peikert}}
\institute{Humboldt{\textendash}Universität zu Berlin}
\date{
\scriptsize {\lolit Slides:} \href{https://github.com/aaronpeikert/ForestEden}{\tt \scriptsize
  \color{foreground} https://github.com/aaronpeikert/ForestEden}
}


\begin{document}

% title slide
{
\setbeamertemplate{footline}{} % no page number here
\frame{
  \titlepage

  \vfill \hfill \includegraphics[height=6mm]{Figs/cc-zero.png} \vspace*{-1cm}
  
  \note{These Slides are for a short Talk (10min) intended vor a beginner audience where no statistical knowledge can be expected. 

    Source: {\tt https://github.com/aaronpeikert/ForestEden}
}}

\begin{frame}[c]{The Great Divide}
  \begin{center}
  \large
  ``There are two kinds of people:\\
  those who divide everything in into two kinds\\
  and those who don’t."
  \end{center}
  \hfill {\textendash} (paraphrasing) \lolit \href{http://hdl.handle.net/2027/mdp.39015032024203?urlappend=\%3Bseq=203}{Robert Benchley}
  \note{
  This epigram captures the unreasonable assumption behind trees that everyhing is a binary class. At the same time a lot of things can be captures in two classes. This fundamental assumption is in all, but the most trivial cases, false. However it proves to be most versatile and is what centrally drives decision trees.
  }
\end{frame}

\begin{frame}[c]{The Great Divide}
  \large
  \onslide<1-|handout:1>\textcolor<5|handout:0>{lolit}{If \textcolor<2,3,4|handout:0>{lolit}{condition}, then \textcolor<3,4|handout:0>{nvhilit}{class}, else \textcolor<4|handout:0>{vhilit}{other class}.} \\
  \onslide<5-|handout:1> \textcolor<handout:0>{nvhilit}{class} and \textcolor<handout:0>{vhilit}{other class} can be again an other condition.
  \note{
  
  }
\end{frame}

\begin{frame}[c]{Garden Eden of Algorythms}
  \large
  \begin{center}
  \only<1>{``With Decision Trees you can have your cake and eat it too."}
  \Large
  \only<2|handout:0>{Transparent}
  \only<3|handout:0>{Easy to understand}
  \only<4|handout:0>{Natural formulation}
  \only<5|handout:0>{Dirt cheap}
  \only<6|handout:0>{High capacity}
  \only<7|handout:0>{High generalizability}
  \only<8|handout:0>{can model (almost) anything}
  \end{center}
  \note{
  Decision Trees have the following very fortunate properties:
  \begin{itemize}
  \item Transparent, all ``paramter" have a direkt meaning
  \item Easy to understand, the tree can be translated in ``If \ldots then \ldots" \textendash Statements
  \item Natural formulation, these statements are in much use even for humans, it is common to capture expert knowledge in those sentences
  \item Dirt cheap in terms of computational costs
  \item High capacity or high generalizability, whatever is more appropriate
  \item Many data generating processes can be fitted, depending on the choosen capacity even all
  \end{itemize}
  }
\end{frame}

\begin{frame}[c]{Summary}

  \begin{enumerate}
  \itemsep12pt
  \item Important point
  \item More important point
  \item Even more important point
  \end{enumerate}

  \note{
    It's always good to include a summary.
}
\end{frame}

\begin{frame}[c]{}

\Large

\scriptsize {\lolit Source:} \href{https://github.com/aaronpeikert/ForestEden}{\tt \scriptsize
  \color{foreground} https://github.com/aaronpeikert/ForestEden} \quad
\includegraphics[height=5mm]{Figs/cc-zero.png}

\vspace{10mm}

\scriptsize {\lolit Slidedesign:} \href{https://github.com/kbroman/Talk_ReproRes}{\tt \scriptsize
  \color{foreground} shamelessly copied from \emph{Karl Broman} \quad}

\vspace{10mm}

\scriptsize {\lolit Github:} \href{https://github.com/aaronpeikert/}{\tt \color{foreground} github.com/aaronpeikert/}

\vspace{10mm}

\scriptsize {\lolit Mail:} \tt aaron.peikert@hu-berlin.de

\note{
  Here's where you can find me, as well as the slides for this talk.
}
\end{frame}

\end{document}
